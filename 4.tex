\section{Fitting with correlated noise in the time domain.}

Suppose that we measure a time-varying noise signal $n(t)$ that is sampled
at time intervals $\Delta t$. We record the value at times $t_k = k\Delta t$, where $k$ ranges
from 0 to $N-1$, and $N$ is the number of samples recorded. Assume that $N$ is an
even number. The function $n(t)$ can be written as a discrete Fourier transform:

$$n(t) = \sum_{m=0}^{N-1}[A_m\cos(m\omega_0 t) + B_m \sin(m\omega_0 t)]$$

where $\omega_0 = \frac{2\pi}{N\Delta t}$. (Note that one may set $B_0 = B_{N/2} = 0$ without loss of generality here, since the sine function will equal zero if $m = 0$, and at any discrete time $t_k = k\Delta t$ if $m = N/2$.) We may collectively refer to the set of $N$ coefficients $A_0...A_{N/2}, B_1...B{N/2-1}$ as $\tilde{n}$.

A common case is Gaussian stationary noise. This can be defined as noise for which, at every frequency $f_m = \frac{m\omega_0}{2\pi}$, $A_m$ and $B_m$ are two indepen-dent random variables distributed as Gaussians with mean of zero and standard deviation of $\sigma_m$. (Of course $B_0 = B_{N/2} = 0$ still.) The quantity $\sigma_m^2$ is proportional to a quantity called the “power spectral density” of the noise. If $\sigma_m$ has a constant value $\sigma$ for all frequencies (that is, $\sigma_m = \sigma$ for all values of $m$), then
$n(t)$ is called ``white noise''.

\todo[inline]{this question has a note!}

\begin{enumerate}[label=\textbf{\Alph*}.]
    \item Consider the measurements $n(t_1)$ and $n(t_2)$ taken at two possibly different
    times $t_1 = k_1\Delta t$ and $t_2 = k_2\Delta t$. Derive a formula for the covariance
    $\operatorname{cov}(n(t_1), n(t_2))$. Calculate the mean and variance of $n(t_k)$.

    \item Suppose we are trying to fit a function $C_s(t)$ to some measured time series, where $s(t)$ is a known shape and $C$ is an unknown normalization we would like to fit for. Our model for the measured data $g(t)$ is $g(t) = C_s(t) + n(t)$, where $n(t)$ is the randomly generated noise from our stationary noise model described above. If we write down a least squares fit directly using the $N$ data points $g(t_k)$, we would find that they have a non-trivial covariance matrix (see Part A). But suppose that we take a discrete Fourier transform of $g(t)$ and $s(t)$ to get some sets of coefficients $\tilde{g}$ and $\tilde{s}$, analogous to $\tilde{n}$. Show that using these you can now write down a much simpler expression for the least squares formula. Do this, and taking its derivative with respect to $C$ and setting it equal to zero, derive a formula for the best fit $\hat{C}$ in terms of $g(t)$, $s(t)$, and $\sigma_m$.

\end{enumerate}
